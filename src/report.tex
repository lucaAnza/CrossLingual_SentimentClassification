\documentclass[11pt]{article}

% ---------- Packages ----------
\usepackage[a4paper,margin=2.5cm]{geometry}
\usepackage{graphicx}
\usepackage{booktabs}
\usepackage{array}
\usepackage{hyperref}
\usepackage{amsmath}
\usepackage{enumitem}
\usepackage{xcolor}
\usepackage{caption}
\usepackage{subcaption}
\usepackage{float}
\usepackage{todonotes}

% ---------- Title ----------
\title{\textbf{Sentiment Analysis of Multilingual Amazon Reviews}\\
AI509/508/DM894 Final Project Report}
\author{Imanshu Sharma \and Alberto Aiello \and Luca Anzaldi}
\date{\today}

\begin{document}
\maketitle

% Optional (not required by template, but commonly useful)
\begin{abstract}
This project investigates sentiment prediction on a multilingual Amazon Reviews dataset.
We study (i) how much target-language data is needed to reach near-peak performance,
(ii) whether a single multilingual model trained on all languages outperforms monolingual
models trained per language, and (iii) the per-language ranking of performance.
We present an initial baseline experiment using a compact multilingual encoder model to
validate the pipeline and assess training efficiency.
\end{abstract}

% =========================================================
% 1. Introduction and Problem Presentation
% =========================================================
\section{Introduction and Problem Presentation}

\subsection{Task description}
Given an Amazon review text, the goal is to predict its sentiment.
We consider two related formulations:
\begin{itemize}[itemsep=2pt]
    \item \textbf{Classification:} predict discrete sentiment labels (e.g., negative / neutral / positive or 1--5 stars).
    \item \textbf{Regression:} predict a continuous sentiment score derived from the 1--5 star rating.
\end{itemize}
In later experiments, we focus on cross-lingual behavior and compare performance across languages.

\subsection{Project goal and research questions}
Our project investigates the following questions:
\begin{enumerate}[itemsep=2pt]
    \item \textbf{How much target-language data is needed?}
    We vary training set size (e.g., 10\%, 30\%, 50\%, 70\%, 100\%) and identify where performance saturates.
    \item \textbf{One multilingual model vs. monolingual models: which works better?}
    We compare a shared multilingual model trained on all languages against separate models trained per language.
    \item \textbf{Per-language performance ranking:}
    We compute and compare evaluation metrics for each language to understand which languages are easier/harder.
\end{enumerate}

\subsection{Why the problem is challenging}
\begin{itemize}[itemsep=2pt]
    \item \textbf{Noisy labels:} star ratings may not perfectly reflect sentiment in the text.
    \item \textbf{Ordinal target:} 1--5 stars are ordered categories; mistakes are not equally severe (1$\rightarrow$2 vs 1$\rightarrow$5).
    \item \textbf{Multilingual variability:} languages differ in script, morphology, and tokenization behavior.
    \item \textbf{Domain shift across categories:} product categories and writing style may vary significantly.
\end{itemize}

% =========================================================
% 2. Method
% =========================================================
\noindent\rule{\linewidth}{0.4pt}
\section{Method}

\subsection{Computing resources (Hardware)}
All experiments were executed on the SDU/University GPU cluster. In the table below are showed the \textit{hardware specification}.

\begin{table}[H]
\centering
\begin{tabular}{ll}
\toprule
\textbf{Component} & \textbf{Specification} \\
\midrule
Compute node & \texttt{u1-gpu-h-1} \\
CPU & Intel Xeon Gold 6230 (20 vCPUs) \\
RAM & 46 GB \\
GPU & 1 $\times$ NVIDIA V100 \\
OS / Environment & Linux + Conda \\
\bottomrule
\end{tabular}
\caption{Hardware configuration used for model training and evaluation.}
\label{tab:hardware}
\end{table}

\subsection{Software stack (Frameworks and tools)}
We implemented the pipeline in Python using the following libraries:
\begin{itemize}[itemsep=2pt]
    \item \textbf{PyTorch}: backend for model training and automatic differentiation.
    \item \textbf{Hugging Face Transformers}: model loading, tokenization, and the training loop (\texttt{Trainer}).
    \item \textbf{Hugging Face Datasets}: loading CSV splits and preprocessing with efficient mapping.
    \item \textbf{Evaluate}: metric computation during evaluation.
    \item \textbf{Weights \& Biases (wandb)}: experiment tracking (loss curves, metrics, run metadata).
    \item \textbf{NumPy}: metric utilities and post-processing.
\end{itemize}

\subsection{Dataset overview}
We use a multilingual Amazon Reviews dataset split into \texttt{train.csv}, \texttt{validation.csv}, and \texttt{test.csv}.
Each review includes a 1--5 star rating and textual fields (title/body), along with metadata such as language and product category.
The dataset covers multiple languages (e.g., EN, ES, DE, FR, JA, ZH).

For the baseline experiments, we subsample the dataset to enable faster experimentation while preserving class balance.
Specifically, we use up to 1.2M training samples and cap validation and test splits at 30k samples each.

\subsubsection{Label distribution}
The training split is perfectly balanced across the five star ratings, as shown in Table~\ref{tab:label_distribution_train}.

\begin{table}[H]
\centering
\begin{tabular}{cc}
\toprule
\textbf{Star rating} & \textbf{Number of samples} \\
\midrule
1 & 240{,}000 \\
2 & 240{,}000 \\
3 & 240{,}000 \\
4 & 240{,}000 \\
5 & 240{,}000 \\
\bottomrule
\end{tabular}
\caption{Label distribution in the training set (balanced).}
\label{tab:label_distribution_train}
\end{table}

The validation and test sets follow the same balanced distribution, with 6{,}000 samples per star rating.

\subsubsection{Dataset split sizes}
Table~\ref{tab:dataset_splits} summarizes the number of samples used in each split.

\begin{table}[H]
\centering
\begin{tabular}{lcc}
\toprule
\textbf{Split} & \textbf{Samples per class} & \textbf{Total samples} \\
\midrule
Training & 240{,}000 & 1{,}200{,}000 \\
Validation & 6{,}000 & 30{,}000 \\
Test & 6{,}000 & 30{,}000 \\
\bottomrule
\end{tabular}
\caption{Overview of dataset splits used in the experiments.}
\label{tab:dataset_splits}
\end{table}

This balanced setup allows a fair comparison between regression and classification approaches
and ensures that evaluation metrics are not biased by class imbalance.

\subsection{Model choice}
We use \texttt{distilbert-base-multilingual-cased}, an \textbf{encoder-only} Transformer model.
It is a distilled variant of multilingual BERT, providing a good trade-off between computational
efficiency and multilingual coverage \cite{sanh2019distilbert,wolf2020transformers}.\begin{table}[H]
\centering
\begin{tabular}{ll}
\toprule
\textbf{Feature} & \textbf{Specification} \\
\midrule
Model name & \texttt{distilbert-base-multilingual-cased} \\
Architecture & Encoder-only Transformer (DistilBERT) \\
Number of layers & 6 Transformer encoder layers \\
Hidden size & 768 \\
Attention heads & 12 \\
Total parameters & $\sim$134 million \\
Vocabulary size & $\sim$119{,}547 \\
Maximum sequence length & 512 tokens \\
Pretraining objective & Masked Language Modeling (MLM) \\
Languages covered & 100+ languages \\
Case sensitive & Yes (cased) \\
Disk size (weights) & $\sim$517 MB \\
\bottomrule
\end{tabular}
\caption{Main characteristics of the \texttt{distilbert-base-multilingual-cased} model \cite{hf_distilbert_multilingual}}
\label{tab:model_features}
\end{table}

\subsection{Preprocessing and input representation}
\paragraph{Column selection.}
We rename \texttt{review\_body} to \texttt{text} and \texttt{stars} to \texttt{label}, removing non-essential metadata
(e.g., IDs and product/category fields) for the baseline.

\paragraph{Tokenization.}
We tokenize the review text with the model tokenizer using truncation and dynamic padding (via \texttt{DataCollatorWithPadding}).
\todo{Add max\_length used (e.g., 256/512) and any title+body concatenation if enabled.}

\subsection{Training setup (Regression on stars)}
We model sentiment as a \textbf{regression} problem where the original star rating
$y \in \{1,2,3,4,5\}$ is normalized to the interval $[0,1]$:
\[
y_{\text{norm}} = \frac{y - 1}{4}.
\]
During evaluation, predictions are mapped back to the original scale for interpretability:
\[
\hat{y} = 4\cdot \hat{y}_{\text{norm}} + 1.
\]

\paragraph{Loss function.}
For regression, \texttt{AutoModelForSequenceClassification} uses Mean Squared Error (MSE) by default.
\todo{If you switch to Huber loss (SmoothL1), mention it here and justify why it is more robust to noisy labels.}

\subsection{Evaluation metrics}
Evaluation is performed using different metrics depending on whether the task is formulated
as \textbf{regression} or \textbf{classification}.

\subsubsection{Regression setting.}
When modeling sentiment as a continuous variable, we report the following metrics:
\begin{itemize}[itemsep=2pt]
    \item \textbf{Mean Absolute Error (MAE)} and \textbf{Root Mean Squared Error (RMSE)}, computed in the original 1--5 star scale, to measure average prediction error.
    \item \textbf{Rounded Accuracy (5-class)}, obtained by rounding the continuous prediction to the nearest integer star rating.
    \item \textbf{Rounded Accuracy (3-class)}, computed after mapping 1--2$\rightarrow$negative, 3$\rightarrow$neutral, and 4--5$\rightarrow$positive.
    \item \textbf{Spearman’s rank correlation coefficient ($\rho$)}, to assess how well the model preserves the ordinal ranking of sentiment.
    \item \textbf{Quadratic Weighted Kappa (QWK)}, which measures ordinal agreement while penalizing larger rating discrepancies more heavily.
\end{itemize}

\subsubsection{Classification setting.}
When sentiment prediction is formulated as a classification task, we report:
\begin{itemize}[itemsep=2pt]
    \item \textbf{Accuracy (5-class)} and \textbf{Macro F1-score (5-class)}, to evaluate exact label prediction performance across all star ratings.
    \item \textbf{Quadratic Weighted Kappa (QWK)}, to account for the ordinal structure of the 1--5 star labels.
    \item \textbf{Accuracy (3-class)} and \textbf{Macro F1-score (3-class)}, computed after mapping ratings to negative, neutral, and positive sentiment.
\end{itemize}


\subsection{Experiment protocol}
\todo{TODO}


% =========================================================
% 3. Results
% =========================================================
\noindent\rule{\linewidth}{0.4pt}
\section{Results}

\subsection{Baseline results (Regression)}
Table~\ref{tab:baseline_results} reports the evaluation metrics for our baseline regression run.
\todo{Replace the example numbers with your final results and specify the split (validation or test).}

\begin{table}[H]
\centering
\begin{tabular}{lcccccc}
\toprule
\textbf{Run} & \textbf{MAE} $\downarrow$ & \textbf{RMSE} $\downarrow$ & \textbf{Acc@5} $\uparrow$ &
\textbf{Acc@3} $\uparrow$ & \textbf{Spearman} $\uparrow$ & \textbf{QWK} $\uparrow$ \\
\midrule
Baseline (example) & 0.53 & 0.72 & 0.59 & 0.77 & 0.84 & 0.85 \\
\bottomrule
\end{tabular}
\caption{Baseline regression results (example values shown). Lower is better for MAE/RMSE; higher is better for the other metrics.}
\label{tab:baseline_results}
\end{table}

\subsection{Qualitative inspection}
\todo{Add 3--5 example reviews with: true stars, predicted stars, and a brief comment on errors (e.g., sarcasm, short text, mixed language).}

\subsection{Planned result visualizations}
For the full experimental grid, we will present:
\begin{itemize}[itemsep=2pt]
    \item a line plot: training size vs performance (one curve per model setup);
    \item per-language bar charts of QWK/RMSE;
    \item a confusion matrix for rounded 5-class predictions (optional).
\end{itemize}

% =========================================================
% 4. Discussion
% =========================================================
\section{Discussion}

\subsection{Key findings}
Our baseline confirms that a multilingual encoder fine-tuned as a regressor can capture sentiment intensity and ordinal structure.
In particular, high Spearman and QWK values indicate strong rank preservation and ordinal agreement, even when exact star prediction is imperfect.
\todo{Rewrite this paragraph using your final numbers and add 1--2 concrete insights.}

\subsection{Interpretation}
\begin{itemize}[itemsep=2pt]
    \item \textbf{Ordinal agreement:} QWK highlights that most errors are ``local'' (e.g., off by 1 star rather than 3+).
    \item \textbf{Cross-lingual behavior:} performance may vary with script and tokenization (Latin vs CJK).
    \item \textbf{Regression vs classification:} regression leverages the ordered nature of stars, while classification may optimize exact label matches.
\end{itemize}

\subsection{Limitations}
\begin{itemize}[itemsep=2pt]
    \item \textbf{Dataset noise and subjectivity:} stars do not always align with textual sentiment.
    \item \textbf{Baseline simplicity:} using only review body may miss useful signal from review title or metadata.
    \item \textbf{Compute constraints:} full multilingual grids are costly; careful sampling strategies are needed.
\end{itemize}

\subsection{Future work}
\begin{itemize}[itemsep=2pt]
    \item Run the full training-size sweep and detect performance plateaus per language.
    \item Compare multilingual vs monolingual training for each language.
    \item Explore robust losses (Huber) and/or ordinal regression objectives.
    \item Add title+body concatenation and test max sequence lengths (256 vs 512).
\end{itemize}

\subsection{Reflection}
If starting over, we would:
\begin{itemize}[itemsep=2pt]
    \item define a unified checkpoint selection rule early (e.g., best QWK with RMSE tie-breaker);
    \item standardize splits and sampling per language from the beginning;
    \item design plots and reporting templates before running the full grid.
\end{itemize}

% =========================================================
% 5. Conclusion
% =========================================================
\section{Conclusion}
We presented a baseline pipeline for multilingual sentiment prediction on Amazon Reviews and validated training efficiency on a GPU cluster.
Using a multilingual DistilBERT encoder, the regression setup achieves strong ordinal agreement (QWK) and rank preservation (Spearman),
supporting the feasibility of larger-scale cross-lingual experiments.
Future work will systematically evaluate data requirements per language and compare multilingual vs monolingual training strategies.

% =========================================================
% References
% =========================================================
\clearpage
\begin{thebibliography}{9}

\bibitem{devlin2019bert}
Devlin J., Chang M.-W., Lee K., Toutanova K.,
\textit{BERT: Pre-training of Deep Bidirectional Transformers for Language Understanding},
\textit{Proceedings of NAACL-HLT}, 2019, pp. 4171--4186.

\bibitem{sanh2019distilbert}
Sanh V., Debut L., Chaumond J., Wolf T.,
\textit{DistilBERT, a distilled version of BERT: smaller, faster, cheaper and lighter},
\textit{arXiv preprint} arXiv:1910.01108, 2019.

\bibitem{wolf2020transformers}
Wolf T., Debut L., Sanh V., et al.,
\textit{Transformers: State-of-the-Art Natural Language Processing},
\textit{Proceedings of EMNLP: System Demonstrations}, 2020, pp. 38--45.

\bibitem{wandb}
Biewald L.,
\textit{Experiment Tracking with Weights and Biases},
\textit{Software available from wandb.com}, 2020.

\bibitem{kaggle_amazon_multilingual}
Mexwell (Kaggle uploader),
\textit{Amazon Reviews Multi (Kaggle dataset)},
\textit{Kaggle}, year unknown.
\url{https://www.kaggle.com/datasets/mexwell/amazon-reviews-multi/data}

\bibitem{hf_distilbert_multilingual}
Hugging Face,
\textit{distilbert-base-multilingual-cased: Model card},
Hugging Face Model Hub, 2019.
Available at: \url{https://huggingface.co/distilbert-base-multilingual-cased}

\end{thebibliography}

% =========================================================
% Appendix A: Reflection with respect to learning objectives
% =========================================================
\appendix
\section{Reflection with respect to learning objectives}
This project aligns with AI509 learning objectives by requiring:
\begin{itemize}[itemsep=2pt]
    \item dataset handling and preprocessing (tokenization, splits, normalization);
    \item Transformer-based modeling (encoder-only architectures, fine-tuning);
    \item training and evaluation methodology (losses, metrics, model selection);
    \item experimental rigor and tracking (reproducibility, W\&B logging);
    \item interpretation of results and limitations in real-world NLP data.
\end{itemize}
\todo{Adapt this to the exact course objectives list from the course description.}

% =========================================================
% Appendix B: Work tasks distribution
% =========================================================
\section{Work tasks distribution}

\begin{table}[H]
\centering
\begin{tabular}{p{4cm}p{11cm}}
\toprule
\textbf{Member} & \textbf{Main contributions} \\
\midrule
Imanshu Sharma & \todo{e.g., dataset exploration, baseline classification runs, plots, report writing} \\
Alberto Aiello & \todo{e.g., experiment design, multilingual/monolingual comparison, evaluation metrics} \\
Luca Anzaldi & \todo{e.g., regression setup, metrics (QWK/Spearman), training pipeline, W\&B tracking} \\
\bottomrule
\end{tabular}
\caption{Work distribution (to be finalized).}
\end{table}

\end{document}
